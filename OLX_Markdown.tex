\documentclass[]{article}
\usepackage{lmodern}
\usepackage{amssymb,amsmath}
\usepackage{ifxetex,ifluatex}
\usepackage{fixltx2e} % provides \textsubscript
\ifnum 0\ifxetex 1\fi\ifluatex 1\fi=0 % if pdftex
  \usepackage[T1]{fontenc}
  \usepackage[utf8]{inputenc}
\else % if luatex or xelatex
  \ifxetex
    \usepackage{mathspec}
  \else
    \usepackage{fontspec}
  \fi
  \defaultfontfeatures{Ligatures=TeX,Scale=MatchLowercase}
\fi
% use upquote if available, for straight quotes in verbatim environments
\IfFileExists{upquote.sty}{\usepackage{upquote}}{}
% use microtype if available
\IfFileExists{microtype.sty}{%
\usepackage{microtype}
\UseMicrotypeSet[protrusion]{basicmath} % disable protrusion for tt fonts
}{}
\usepackage[margin=1in]{geometry}
\usepackage{hyperref}
\hypersetup{unicode=true,
            pdftitle={OLX Project},
            pdfauthor={Mo'men Mohamed},
            pdfborder={0 0 0},
            breaklinks=true}
\urlstyle{same}  % don't use monospace font for urls
\usepackage{color}
\usepackage{fancyvrb}
\newcommand{\VerbBar}{|}
\newcommand{\VERB}{\Verb[commandchars=\\\{\}]}
\DefineVerbatimEnvironment{Highlighting}{Verbatim}{commandchars=\\\{\}}
% Add ',fontsize=\small' for more characters per line
\usepackage{framed}
\definecolor{shadecolor}{RGB}{248,248,248}
\newenvironment{Shaded}{\begin{snugshade}}{\end{snugshade}}
\newcommand{\KeywordTok}[1]{\textcolor[rgb]{0.13,0.29,0.53}{\textbf{#1}}}
\newcommand{\DataTypeTok}[1]{\textcolor[rgb]{0.13,0.29,0.53}{#1}}
\newcommand{\DecValTok}[1]{\textcolor[rgb]{0.00,0.00,0.81}{#1}}
\newcommand{\BaseNTok}[1]{\textcolor[rgb]{0.00,0.00,0.81}{#1}}
\newcommand{\FloatTok}[1]{\textcolor[rgb]{0.00,0.00,0.81}{#1}}
\newcommand{\ConstantTok}[1]{\textcolor[rgb]{0.00,0.00,0.00}{#1}}
\newcommand{\CharTok}[1]{\textcolor[rgb]{0.31,0.60,0.02}{#1}}
\newcommand{\SpecialCharTok}[1]{\textcolor[rgb]{0.00,0.00,0.00}{#1}}
\newcommand{\StringTok}[1]{\textcolor[rgb]{0.31,0.60,0.02}{#1}}
\newcommand{\VerbatimStringTok}[1]{\textcolor[rgb]{0.31,0.60,0.02}{#1}}
\newcommand{\SpecialStringTok}[1]{\textcolor[rgb]{0.31,0.60,0.02}{#1}}
\newcommand{\ImportTok}[1]{#1}
\newcommand{\CommentTok}[1]{\textcolor[rgb]{0.56,0.35,0.01}{\textit{#1}}}
\newcommand{\DocumentationTok}[1]{\textcolor[rgb]{0.56,0.35,0.01}{\textbf{\textit{#1}}}}
\newcommand{\AnnotationTok}[1]{\textcolor[rgb]{0.56,0.35,0.01}{\textbf{\textit{#1}}}}
\newcommand{\CommentVarTok}[1]{\textcolor[rgb]{0.56,0.35,0.01}{\textbf{\textit{#1}}}}
\newcommand{\OtherTok}[1]{\textcolor[rgb]{0.56,0.35,0.01}{#1}}
\newcommand{\FunctionTok}[1]{\textcolor[rgb]{0.00,0.00,0.00}{#1}}
\newcommand{\VariableTok}[1]{\textcolor[rgb]{0.00,0.00,0.00}{#1}}
\newcommand{\ControlFlowTok}[1]{\textcolor[rgb]{0.13,0.29,0.53}{\textbf{#1}}}
\newcommand{\OperatorTok}[1]{\textcolor[rgb]{0.81,0.36,0.00}{\textbf{#1}}}
\newcommand{\BuiltInTok}[1]{#1}
\newcommand{\ExtensionTok}[1]{#1}
\newcommand{\PreprocessorTok}[1]{\textcolor[rgb]{0.56,0.35,0.01}{\textit{#1}}}
\newcommand{\AttributeTok}[1]{\textcolor[rgb]{0.77,0.63,0.00}{#1}}
\newcommand{\RegionMarkerTok}[1]{#1}
\newcommand{\InformationTok}[1]{\textcolor[rgb]{0.56,0.35,0.01}{\textbf{\textit{#1}}}}
\newcommand{\WarningTok}[1]{\textcolor[rgb]{0.56,0.35,0.01}{\textbf{\textit{#1}}}}
\newcommand{\AlertTok}[1]{\textcolor[rgb]{0.94,0.16,0.16}{#1}}
\newcommand{\ErrorTok}[1]{\textcolor[rgb]{0.64,0.00,0.00}{\textbf{#1}}}
\newcommand{\NormalTok}[1]{#1}
\usepackage{graphicx,grffile}
\makeatletter
\def\maxwidth{\ifdim\Gin@nat@width>\linewidth\linewidth\else\Gin@nat@width\fi}
\def\maxheight{\ifdim\Gin@nat@height>\textheight\textheight\else\Gin@nat@height\fi}
\makeatother
% Scale images if necessary, so that they will not overflow the page
% margins by default, and it is still possible to overwrite the defaults
% using explicit options in \includegraphics[width, height, ...]{}
\setkeys{Gin}{width=\maxwidth,height=\maxheight,keepaspectratio}
\IfFileExists{parskip.sty}{%
\usepackage{parskip}
}{% else
\setlength{\parindent}{0pt}
\setlength{\parskip}{6pt plus 2pt minus 1pt}
}
\setlength{\emergencystretch}{3em}  % prevent overfull lines
\providecommand{\tightlist}{%
  \setlength{\itemsep}{0pt}\setlength{\parskip}{0pt}}
\setcounter{secnumdepth}{0}
% Redefines (sub)paragraphs to behave more like sections
\ifx\paragraph\undefined\else
\let\oldparagraph\paragraph
\renewcommand{\paragraph}[1]{\oldparagraph{#1}\mbox{}}
\fi
\ifx\subparagraph\undefined\else
\let\oldsubparagraph\subparagraph
\renewcommand{\subparagraph}[1]{\oldsubparagraph{#1}\mbox{}}
\fi

%%% Use protect on footnotes to avoid problems with footnotes in titles
\let\rmarkdownfootnote\footnote%
\def\footnote{\protect\rmarkdownfootnote}

%%% Change title format to be more compact
\usepackage{titling}

% Create subtitle command for use in maketitle
\newcommand{\subtitle}[1]{
  \posttitle{
    \begin{center}\large#1\end{center}
    }
}

\setlength{\droptitle}{-2em}

  \title{OLX Project}
    \pretitle{\vspace{\droptitle}\centering\huge}
  \posttitle{\par}
    \author{Mo'men Mohamed}
    \preauthor{\centering\large\emph}
  \postauthor{\par}
      \predate{\centering\large\emph}
  \postdate{\par}
    \date{December 12, 2018}


\begin{document}
\maketitle

\subsubsection{- What is the aim from this Project
:}\label{what-is-the-aim-from-this-project}

getting all information related to real state prices in alexandria

this information are from \href{https://olx.com.eg/en/}{\textbf{OLX}}
which is a site for selling and bying every thing online

\subsubsection{- to whome this information are going to be useful
?}\label{to-whome-this-information-are-going-to-be-useful}

to all who are interested in selling or buying real states

whether they are a companies, individuals Or Even middlemen

\begin{center}\rule{0.5\linewidth}{\linethickness}\end{center}

\subsection{\texorpdfstring{\emph{Analysis
statring}}{Analysis statring}}\label{analysis-statring}

first of all loading all packges we need:

\begin{Shaded}
\begin{Highlighting}[]
\KeywordTok{library}\NormalTok{(dplyr)}
\KeywordTok{library}\NormalTok{(tidyr)}
\KeywordTok{library}\NormalTok{(magrittr)}
\end{Highlighting}
\end{Shaded}

We will start our scrabing from this
\href{https://olx.com.eg/en/properties/properties-for-sale/apartments-for-sale/alexandria/}{\emph{url}}
which is the url of real state for selling in alex only

\begin{Shaded}
\begin{Highlighting}[]
\NormalTok{url<-}\StringTok{'https://olx.com.eg/en/properties/properties-for-sale/apartments-for-sale/alexandria/'}
\end{Highlighting}
\end{Shaded}

after that we will getting all \emph{css} and \emph{xpath} we need to
collect \textbf{basic information} from the url

\begin{Shaded}
\begin{Highlighting}[]
\NormalTok{title_css<-}\StringTok{'.ads__item__title'}

\NormalTok{location_css<-}\StringTok{ '.ads__item__location'}

\NormalTok{price_css<-}\StringTok{'.price'}
\end{Highlighting}
\end{Shaded}

then we will find that the web site contains 500 pages having the
information we need, so we will generate an object by loop contains all
pages which are would to be scrabed

\begin{Shaded}
\begin{Highlighting}[]
\NormalTok{urls<-}\KeywordTok{list}\NormalTok{()}

\ControlFlowTok{for}\NormalTok{( i }\ControlFlowTok{in} \DecValTok{1}\OperatorTok{:}\DecValTok{500}\NormalTok{)\{}
\NormalTok{        urls[[i]]<-}\KeywordTok{paste0}\NormalTok{(}\StringTok{'https://olx.com.eg/en/properties/properties-for-sale/apartments-for-sale/alexandria/?page='}\NormalTok{,i)}
\NormalTok{\}}
\end{Highlighting}
\end{Shaded}

The Next code is to scrab the basic information

but we will not excute it to save time

\begin{Shaded}
\begin{Highlighting}[]
\NormalTok{olx_ads<-}\KeywordTok{list}\NormalTok{()}

\ControlFlowTok{for}\NormalTok{( i }\ControlFlowTok{in} \DecValTok{1}\OperatorTok{:}\KeywordTok{length}\NormalTok{(urls))\{}
\NormalTok{        olx<-}\KeywordTok{read_html}\NormalTok{(}\DataTypeTok{x =}\NormalTok{ urls[[i]])}
\NormalTok{        title<-olx}\OperatorTok\KeywordTok{html_nodes}\NormalTok{(}\DataTypeTok{css =}\NormalTok{ title_css)}\OperatorTok\KeywordTok{html_text}\NormalTok{(}\DataTypeTok{trim =}\NormalTok{ T)}
\NormalTok{        location<-olx}\OperatorTok\KeywordTok{html_nodes}\NormalTok{(}\DataTypeTok{css =}\NormalTok{ location_css)}\OperatorTok\KeywordTok{html_text}\NormalTok{(}\DataTypeTok{trim =}\NormalTok{ T)}
\NormalTok{        price<-olx}\OperatorTok\KeywordTok{html_nodes}\NormalTok{(}\DataTypeTok{css =}\NormalTok{ price_css)}\OperatorTok\KeywordTok{html_text}\NormalTok{(}\DataTypeTok{trim =}\NormalTok{ T)}
\NormalTok{        price<-price[}\OperatorTok{-}\KeywordTok{c}\NormalTok{(}\DecValTok{1}\NormalTok{,}\DecValTok{2}\NormalTok{)]}
\NormalTok{        url<-olx}\OperatorTok\KeywordTok{html_nodes}\NormalTok{(}\DataTypeTok{css =}\NormalTok{ title_css)}\OperatorTok\KeywordTok{html_attr}\NormalTok{(}\StringTok{'href'}\NormalTok{)}
\NormalTok{        olx_ads[[i]]<-}\KeywordTok{data.frame}\NormalTok{(title,location,price,url)}
        \KeywordTok{gc}\NormalTok{()}
\NormalTok{\}}
\end{Highlighting}
\end{Shaded}

this code will give us an object called \textbf{olx\_ads}

then we need to convert it to \emph{data frame} using this code to bind
all rows in the list

\begin{Shaded}
\begin{Highlighting}[]
\NormalTok{olx_ads<-}\KeywordTok{bind_rows}\NormalTok{(olx_ads)}
\end{Highlighting}
\end{Shaded}

after looking to the data we will find that there are alot of ads
repeation

so we need to remove duplicated record

\begin{Shaded}
\begin{Highlighting}[]
\NormalTok{olx_ads<-}\KeywordTok{distinct}\NormalTok{(olx_ads)}
\end{Highlighting}
\end{Shaded}


\end{document}
